%
% Halaman Nota pembimbing
%
% @author  Slamet B. Aryo
% @version 1.10
%

\chapter*{NOTA PEMBIMBING}

\vspace*{0.2cm}

\begin{spacing}{1}
\noindent
\bo{\pembimbingSatu}\\
Alamat dosen pembimbing\\
\underline{Alamat baris kedua jika ada}\\

\noindent
\begin{tabbing}
Lamp. \hspace{1em}  \= : 2 (dua) eksemplar\\
Hal 				\> : Naskah Skripsi Sdr. \penulis
\end{tabbing}

\noindent
Kepada Yth.\\
Dekan Fakultas \fakultas\\
\textit{c.q.} Ketua Jurusan \program\\
\underline{PEKALONGAN}\\[1em]
\end{spacing}

\begin{spacing}{1.215}

\noindent
\emph{Assalamualaikum Wr. Wb.}\\

\noindent
Setelah diadakan penelitian dan perbaikan seperlunya,
maka bersama ini saya kirimkan naskah \text{\type} saudara:\\

\noindent
\begin{tabularx}{\linewidth}{@{}l c @{\hspace{.5em}}X@{}}
	Nama &: & \bo{\penulis} \\
	NIM &: & \bo{\nim} \\
	Judul \Type &: & \bo{\judul} \\
\end{tabularx} \\

\vspace*{1em}

\noindent
Naskah tersebut sudah memenuhi persyaratan untuk dapat segera dimunaqosahkan.
Demikian nota pembimbing ini dibuat untuk digunakan sebagaimana mestinya.
Atas perhatiannya, saya sampaikan terima kasih.\\[0.3cm]

\noindent
\emph{Wassalamualaikum Wr. Wb.}

\begin{ttdkanan}
Pekalongan, \tanggalPengesahan \\
Pembimbing,\\[3em]

\noindent
\underline{\bo{\pembimbingSatu}}\\[0.1cm]
NIP. \nipPembimbingSatu
\end{ttdkanan}

\end{spacing}

\newpage
