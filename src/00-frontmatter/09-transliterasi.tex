%
% Halaman Transliterasi Arab-Latin
%
% @author  Slamet B. Aryo
% @version 1.01
%
\chapter*{Pedoman Transliterasi Arab-Latin}

% \begin{center}
%  \bo{Sesuai dengan SKB Menteri Agama dan Menteri Pendidikan dan Kebudayaan RI
%  No. 158/1987 dan No. 0543b/U/1987.\\
%  Tertanggal 12 Januari 1988}\\[2em]
%
%  Yang dapat diakses pada tautan:\\
%  http://bit.ly/TRANSLITERASI\_ArabLatin
% \end{center}

Transliterasi kata-kata yang dipakai dalam penyusunan skripsi ini berpedoman
pada Surat Keputusan Bersama antara Menteri Agama Nomor: 158/1987 dan Menteri Pendidikan dan Kebudayaan RI Nomor: 0543b/U/1987.

\textbf{Konsonan}
% \vspace*{1em}

{\renewcommand{\arraystretch}{1.12}
\begin{table}[h]
\centering
\begin{tabular}{|c|c|c|c|}
\hline
\textbf{Huruf Arab} & \textbf{Nama} & \textbf{Huruf Latin} & \textbf{Keterangan}   \\ \hline
\textarabic{ا}        & alif            & tidak dilambangkan  &  tidak dilambangkan     \\[16pt] \hline
\textarabic{ب}        & ba'             & B                & Be                    \\[16pt] \hline
\textarabic{ت}        & ta'             & T                & Te                    \\[16pt] \hline
\textarabic{ث}        & \ardot{s}a'     & \ardot{S}        & eS (dengan titik di atas)  \\[16pt] \hline
\textarabic{ج}        & jim             & J                & Je                    \\[16pt] \hline
\textarabic{ح}        & \d{h}a'         & \d{H}            & Ha (dengan titik di bawah)  \\[16pt] \hline
\textarabic{خ}        & kha'            & Kh               & Ka dan Ha             \\[16pt] \hline
\textarabic{د}        & dal             & D                & De                    \\[16pt] \hline
\textarabic{ذ}        & \ardot{z}al     & \ardot{Z}        & Zet (dengan titik di atas)  \\[16pt] \hline
\textarabic{ر}        & ra'             & R                & eR                    \\[16pt] \hline
\textarabic{ز}        & zai             & Z                & Zet                   \\[16pt] \hline
\textarabic{س}        & sin             & S                & eS                    \\[16pt] \hline
\textarabic{ش}        & syin            & Sy               & eS dan Ye             \\[16pt] \hline
\textarabic{ص}        & \d{s}ad         & \d{S}            & eS (dengan titik di bawah)  \\[16pt] \hline
\textarabic{ض}        & \d{d}ad         & \d{D}            & De (dengan titik di bawah)  \\[16pt] \hline
\textarabic{ط}        & \d{t}a'         & \d{T}            & Te (dengan titik di bawah)  \\[16pt] \hline
\textarabic{ظ}        & \d{z}a'         & \d{Z}            & Zet (dengan titik di bawah) \\[16pt] \hline
\textarabic{ع}        & `ain            & `                & koma terbalik di atas       \\[16pt] \hline
\textarabic{غ}        & gain            & G                & Ge                    \\[16pt] \hline
\textarabic{ف}        & fa'             & F                & eF                    \\[16pt] \hline
\textarabic{ق}        & qaf             & Q                & Qi                    \\[16pt] \hline
\textarabic{ك}        & kaf             & K                & Ka                    \\[16pt] \hline
\textarabic{ل}        & lam             & L                & eL                    \\[16pt] \hline
\textarabic{م}        & mim             & M                & eM                    \\[16pt] \hline
\textarabic{ن}        & nun             & N                & eN                    \\[16pt] \hline
\textarabic{و}        & waw             & W                & We                    \\[16pt] \hline
\textarabic{‫ه‬}        & ha'             & H                & Ha                    \\[16pt] \hline
\textarabic{ء}        & hamzah          & '                & apostrof              \\[16pt] \hline
\textarabic{ي}        & ya'             & Y                & Ye                    \\[16pt] \hline
\end{tabular}
\end{table}
}

\textbf{Vokal}

{\renewcommand{\arraystretch}{1.5}
\begin{table}[h]
\centering
\begin{tabular}{|c|c|c|c|}
\hline
\textbf{Vokal Tunggal} & \textbf{Vokal Rangkap} & \textbf{Vokal Panjang / Maddah} \\ \hline
\textarabic{أَ} = a     &                   &  \textarabic{ىـٰـــ / ـَا / آ} = \={a} \\ \hline
\textarabic{إِ} = i     & \textarabic{أَيْ} = ai  &  \textarabic{إِيْ / ىـٖى} = \={i}  \\ \hline
\textarabic{أُ} = u     & \textarabic{أَوْ} = au  &  \textarabic{ــُوْ} = \={u}  \\ \hline
\end{tabular}
\end{table}
}

\textbf{Ta' Marbu\d{t}ah}

Transliterasi untuk ta' marbu\d{t}ah ada dua:

\begin{enumerate}
\item Ta' marbu\d{t}ah hidup atau mendapat harakat fat\d{h}ah, kasrah dan
\d{d}ammah, transliterasinya adalah /t/.

\item Ta' marbu\d{t}ah mati atau mendapat harakat sukun, transliterasinya
adalah /h/.
\end{enumerate}


Jika pada kata terakhir dengan ta' marbu\d{t}ah diikuti oleh kata yang
menggunakan kata sandang \textit{al} serta bacaan kedua kata itu terpisah, maka
ta' marbu\d{t}ah itu ditransliterasikan dengan ha (h).

Contoh:

\textarabic{رَوضَةُ الأَ طْفَالُ} = rau\d{d}ah al-a\d{t}f\={a}l / rau\d{d}atula\d{t}f\={a}l

\textarabic{المَدِينَةُ المُنَوَّرَةٌ} = al-Mad\={i}nah al-Munawwarah / al-Mad\={i}natul-Munawwarah

\textarabic{طَلْحَةْ} = \d{t}al\d{h}ah

\textbf{Syaddad (Tasydid)}

Tanda tasydid dilambangkan dengan huruf yang sama dengan huruf
yang diberi tanda syaddad tersebut.

Contoh:

\textarabic{رَبَّنَا} = rabban\={a}

\textarabic{الحَجّ} = al-\d{h}ajj


\textbf{Kata Sandang (Artikel)}

Kata sandang dalam tulisan Arab dilambangkan dengan huruf, yaitu \textarabic{‫ال‬}
namun dalam transliterasi ini kata sandang itu dibedakan menjadi dua:

\begin{enumerate}
\item Kata sandang yang diikuti oleh huruf syamsiyah, ditransliterasikan dengan
bunyinya, yaitu huruf /l/ diganti dengan huruf yang sama dengan huruf yang
langsung mengikuti kata sandang itu.

\item Kata sandang yang diikuti oleh huruf qamariyah, ditransliterasikan sesuai
aturan yang digariskan di depan dan sesuai dengan bunyinya.
\end{enumerate}

Contoh:

\textarabic{الرّجُل} = ar-rajulu

\textarabic{الشّمْس} = asy-syamsu

\textarabic{السّيّدة} = as-sayyidah

\textbf{Hamzah}

Hamzah yang berada di awal kata tidak ditransliterasikan. Akan tetapi,
jika hamzah tersebut berada di tengah kata atau di akhir kata, huruf hamzah itu
ditransliterasikan dengan apostrof /'/

Contoh:

\textarabic{أُمرت} = umirtu

\textarabic{شيءٌ} = syai'un
