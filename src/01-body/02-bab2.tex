%-----------------------------------------------------------------------------%
\chapter{\babDua}
%-----------------------------------------------------------------------------%
Untuk memulai penelitian, dibutuhkan kerangka berpikir yang sesuai untuk permasalahan yang ingin dipecahkan. Untuk membentuk kerangka berpikir yang sesuai, perlu dikaitkan dengan hasil studi literatur yang telah dilakukan. Oleh karena itu, pada bab ini, akan dijelaskan hasil studi literatur yang telah dilakukan yang telah dikaitan dengan kerangka kerja untuk penelitian ini.


%-----------------------------------------------------------------------------%
\section{\f{Sample Theory}}
%-----------------------------------------------------------------------------%
Teori yang saya gunakan dalam penelitian ini adalah teori dari \textcite{Williams2002organizational}.

\begin{figure}
	\centering
	\caption{Pendukung teori yang dipakai}
	\includegraphics[width=0.6\textwidth]{assets/pics/makara.png}\\
	{\footnotesize Sumber: \textcite{Sugiyono2008}\par}
	\label{fig:sample}
\end{figure}

Bahwa \lipsum[34]

Jelaskan \pic~\ref{fig:sample}.

Oleh karena itu, \lipsum[35-37]

%-----------------------------------------------------------------------------%
\section{Keterkaitan Teori Dengan Penelitian}
%-----------------------------------------------------------------------------%
\begin{figure}
	\centering
	\caption{Keterkaitan konsep hasil studi literatur terhadap penelitian}
	\includegraphics[width=0.8\textwidth]{assets/pics/makara.png}\\
	\label{fig:research_concept_map}
	{\footnotesize Sumber: \textcite{Sugiyono2008}\par}
\end{figure}

Jelaskan \pic~\ref{fig:research_concept_map} di sini.
