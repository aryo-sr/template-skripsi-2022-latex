%-----------------------------------------------------------------------------%
\chapter{\babSatu}
%-----------------------------------------------------------------------------%
Pada bab ini, akan dijelaskan tentang latar belakang dan permasalahan yang diselesaikan pada penelitian ini.


%-----------------------------------------------------------------------------%
\section{Latar Belakang}
%-----------------------------------------------------------------------------%
Tentukan latar belakang dari penelitian Anda di sini (\f{background}).
Karena kita berkuliah di perguruan tinggi Islam, maka sepantasnya ada minimal
satu dalil yang dimasukkan, entah itu ayat Al-Quran ataupun \d{h}adi\ardot{s}.
Contohnya sebagai berikut:

\setRTL
\noindent \textarabic{\quranayah*[5][2]} %cukup memasukkan nomor surat dan ayat
\setLTR
% \end{arabic}
\vspace*{12pt}

\begin{spacing}{1.125}
``Wahai orang-orang yang beriman, janganlah kamu melanggar syiar-syiar (kesucian) Allah,
jangan (melanggar kehormatan) bulan-bulan haram, jangan (mengganggu) \textit{hadyu} (hewan-hewan kurban)
dan \textit{qal\=a'id} (hewan-hewan kurban yang diberi tanda), dan jangan (pula mengganggu)
para pengunjung Baitulharam sedangkan mereka mencari karunia dan rida Tuhannya! Apabila kamu
telah bertahalul (menyelesaikan ihram), berburulah (jika mau). Janganlah sekali-kali kebencian(-mu)
kepada suatu kaum, karena mereka menghalang-halangimu dari Masjidilharam, mendorongmu berbuat
melampaui batas (kepada mereka). Tolong-menolonglah kamu dalam (mengerjakan) kebajikan dan takwa,
dan jangan tolong-menolong dalam berbuat dosa dan permusuhan.
Bertakwalah kepada Allah, sesungguhnya Allah sangat berat siksaan-Nya.''
(Al-M\={a}'idah [5]: 2)\par
\end{spacing}
\vspace*{6pt}

\lipsum[44]

%-----------------------------------------------------------------------------%
\section{Permasalahan}
%-----------------------------------------------------------------------------%
Sebutkan permasalahan penelitian Anda dari latar belakang tersebut.

%-----------------------------------------------------------------------------%
\subsection{Rumusan Masalah}
%-----------------------------------------------------------------------------%
Berikut ini adalah rumusan permasalahan dari penelitian yang dilakukan:
\begin{enumerate}
\item Bagaimana cara membuat pertanyaan penelitian?

\item Seperti apa pertanyaan yang bagus?

\item Bagimana cara menentukan pertanyaan yang tepat?
\end{enumerate}


%-----------------------------------------------------------------------------%
\subsection{Batasan Permasalahan}
%-----------------------------------------------------------------------------%
Berikut ini adalah asumsi yang digunakan sebagai batasan penelitian ini:
\begin{enumerate}
	\item Salah satu batasannya adalah, ini hanya \f{template}.
\end{enumerate}


%-----------------------------------------------------------------------------%
\section{Tujuan Penelitian}
%-----------------------------------------------------------------------------%
Berikut ini adalah tujuan penelitian yang dilakukan:
\begin{enumerate}
	\item Untuk memberikan \f{template} yang dapat mempermudah skripsi orang lain.
\end{enumerate}


%-----------------------------------------------------------------------------%
\section{Posisi Penelitian}
%-----------------------------------------------------------------------------%
Sebutkan posisi penelitian Anda.

Pertama, skripsi \citefirstlastauthor{Hani2022} \parencite*{Hani2022} berjudul
``\citefield{Hani2022}{title}'' yang menganalisis\ldots
\lipsum[11]

\begin{figure}
	\centering
	\caption{Penjelasan singkat terkait gambar.}
	\includegraphics[width=\textwidth]{assets/pics/makara.png}\\
	{\footnotesize Sumber: \textcite{Sugiyono2008}\par}
	\label{fig:research_position}
\end{figure}

Jelaskan \pic~\ref{fig:research_position} di sini.


%-----------------------------------------------------------------------------%
\section{Langkah Penelitian}
%-----------------------------------------------------------------------------%
Berikut ini adalah langkah penelitian yang telah dilakukan:
\begin{enumerate}
	\item Tinjauan literatur \\
	Pada tahap ini, dipelajari teori-teori yang terkait dengan penelitian ini untuk mendapatkan konsep dasar yang dibutuhkan dalam mencapai tujuan penelitian.
	\item Analisis implementasi dan kesimpulan \\
	Pada tahap ini, digunakan studi kasus untuk analisis terkait kegunaan \f{template}. Setelah melakukan analisis tersebut, ditarik kesimpulan keseluruhan dari penelitian ini.
\end{enumerate}


%-----------------------------------------------------------------------------%
\section{Sistematika Penulisan}
%-----------------------------------------------------------------------------%
Sistematika penulisan laporan adalah sebagai berikut:
\begin{description}[style=unboxed,leftmargin=2.5cm]
% \begin{itemize}
	\item [Bab 1:] \babSatu \\
	    Bab ini mencakup latar belakang, cakupan penelitian, dan pendefinisian masalah.
	\item [Bab 2:] \babDua \\
	    Bab ini mencakup pemaparan terminologi dan teori yang terkait dengan penelitian berdasarkan hasil tinjauan pustaka yang telah digunakan, sekaligus memperlihatkan kaitan teori dengan penelitian.
	\item [Bab 3:] \babTiga \\
	    Apa itu Bab 3?
	\item [Bab 4:] \babEmpat \\
		Apa itu Bab 4?
	\item [Bab 5:] \babLima, jika ada \\
	    Apa itu Bab 5?
	\item [Bab 6:] \kesimpulan \\
	    Bab ini mencakup kesimpulan akhir penelitian dan saran untuk pengembangan berikutnya.
% \end{itemize}
\end{description}
